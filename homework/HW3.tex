\documentclass[11pt]{article}
\usepackage{url}
\usepackage{alltt}
\usepackage{bm}
\linespread{1}
\textwidth 6.5in
\oddsidemargin 0.in
\addtolength{\topmargin}{-1in}
\addtolength{\textheight}{2in}

\usepackage{amsmath}
\usepackage{amssymb}
\usepackage{bm}

\begin{document}


\begin{center}
\Large
STA 712 Homework 3\\
\normalsize
\vspace{5mm}
\end{center}

\noindent \textbf{Due:} Friday, September 23, 12:00pm (noon) on Canvas.\\ 

\noindent \textbf{Instructions:} Submit your work as a single PDF. For this assignment, you may include written work by scanning it and incorporating it into the PDF. Include all R code needed to reproduce your results in your submission.

\section*{Multivariate normal distributions}

The multivariate normal distribution appears frequently in 712, for example as the asymptotic distribution of our coefficient estimates $\widehat{\beta}$. The purpose of this section is to derive a basic property of the multivariate normal distribution that we use regularly, for example in constructing our Wald test statistic.\\


\noindent One way to define a multivariate normal distribution is with its \textit{moment generating function} (MGF).  Let $X \in \mathbb{R}^k$ be a random vector. The (multivariate) moment generating function $M_X(t)$ of $X$ is defined by
$$M_X(t) = \mathbb{E}[e^{t^T X}],$$
where $t \in \mathbb{R}^k$. As with univariate MGFs, if $M_X(t) = M_Y(t)$ for all $t$, then the two random variables $X$ and $Y$ have the same distribution.\\

\noindent We say that the random vector $X \in \mathbb{R}^k$ follows a multivariate normal distribution with mean $\mu \in \mathbb{R}^k$ and variance matrix $\Sigma \in \mathbb{R}^{k \times k}$, and write $X \sim N(\mu, \Sigma)$, if 
$$M_X(t) = e^{t^T \mu} e^{\frac{1}{2} t^T \Sigma t}.$$

\begin{enumerate}
\item An important property of multivariate normal random variables is that if $X \sim N(\mu, \Sigma)$, then
$$\bm{a} + \bm{B} X \sim N(\bm{a} + \bm{B} \mu, \bm{B} \Sigma \bm{B}^T),$$
where $\bm{a} \in \mathbb{R}^k$ and $\bm{B} \in \mathbb{R}^{m \times k}$. Our goal is to use MGFs to prove this property.

\begin{enumerate}
\item Show that for any random vector $X$ in $\mathbb{R}^k$, the MGF of $Y = \bm{a} + \bm{B} X$ is given by
$$M_Y(t) = e^{t^T \bm{a}} M_X(\bm{B}^T t).$$

\item Using (a), show that if $X \sim N(\mu, \Sigma)$, then $\bm{a} + \bm{B} X \sim N(\bm{a} + \bm{B} \mu, \bm{B} \Sigma \bm{B}^T)$.
\end{enumerate}
\end{enumerate}

\section*{Likelihood ratio tests}

Suppose that 


\section*{Multicollinearity and power}

One of the potential issues with multicollinearity is that it 

\section*{Data analysis}

\end{document}
